% Copyright (c) 2014,2016 Casper Ti. Vector
% Public domain.

\specialchap{序言}
%\pkuthssffaq % 中文测试文字。
	在做这个项目的开发过程中遇到了一系列的问题,在探索这些问题的过程中,自己也从对很多知识的一知半解到熟练使用,比如git版本控制、服务器上的各种权限问题、开源项目的编译配置、makefile的编写以及c和java语言的编译链接问题等等。在学以致用的过程中,总是充满了未知的挑战,很多时候会在某个问题卡住很久,但是当解决了一个个棘手的问题之后,内心的喜悦会证明之前的付出都是值得的。做毕业设计是一个不断探索未知,不断挑战自己的过程,在这个过程中不仅我的研究能力得到了锻炼,而且处理问题的能力也得到了很大提升。

	本文将由两个部分构成:
	\begin{itemize}
	\item \textbf{KVF}:
		%其中根据学校的格式规范\mbox{\supercite{pku-thesisstyle}}%
		该部分描述了KVF的基本原理和设计思路
		主要包括KVF的数据模型、系统架构、应用场景
		此外也介绍了一些基本的API
		
	\item \textbf{Benchmark的实现}:
		该部分介绍如何实现Benchmark
		主要内容包括如何依托开源项目YCSB(Yahoo! Cloud Serving Benchmark)实现Benchmark以及一些基本的测试结果与分析

	\end{itemize}
% vim:ts=4:sw=4
